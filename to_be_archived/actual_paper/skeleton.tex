Introduction:
- What problem are we trying to solve? Intrinsically multiscale, ethical and financial issues
- Many different ways to think about maximally informative experiments
- In practice, power seems like a practical definition.
- Idea that power developes unwanted phase dependence at high frequencies
- How do you control this rigorously/systematically?
- Question becomes, how close can you get to the optimal choice using this kind of technique?

FIG: Show unwanted power oscillations, resolution using staggered grid, comparison of several staggered grids

- In this paper we present an algorithm for selecting optimal measurement collection times for rhythm detection. By rhythm detection we mean cosinor-based rhythmometry where frequency parameter has for instance been estimated from spectral analysis
- We assume that there is uncertainty in frequency and phase, and for this reason seak designs that perform well for several possible frequencies and phases.
- Using results from optimization theory, we reduce problem to a simpler mathematical form.
- Using simplified form, we are able to effectively survey various regimes relevant to sampling and experiments. Understanding of when there is/isn't something worth optimizing.
- One may wonder why we are ignoring effect size? This comes from an analytical expression which shows that effect size decopules from the other harmonic features.
- i.e. if you do better at one effect size, you automatically do better at all other ones

- Some discussion of advanced topics, what did we not cover, bigger context, zoom back out
- What does this paper contain


Methods:

- Harmonic regression model, assumptions 
- Exact expression for power analysis
- How do you read the MedStat result, amplitude decouples letting you focus on non-centrality
- Fisher information matrix
- Reduced Fisher information matrix

Results 1: Optimization framework
- Harmonic regression is not the same thing as what we are doing
- Their result about uniform designs is over-simplified
- State our main result: you can optimize power for a known frequency and budget for all acrophases at once, regardless of amplitude (effect size)
- Easy to extend this to a window of frequencies, choice of augmented cost function

FIG: Comparison of uniform and optimal design for a frequency, and frequency window

- Numerically what do you do? You can use either gradient-based for speed, or convex programming for proving theoretical optimum.
- Given that we are working with 2x2 design, easy to simplify eigenvalue formula. Since it's harmonic regression, trace is fixed and you can reduce eigenvalue optimization to D-optimality.
- Simple linear algebra lets you rewrite this as something close to a quadratic programming problem
- Disciplined convex programming can then handle that. CVXR + Gurobi
- We provide code for doing this in R


Results 2: Application to ultradian rhythm detection
- Consider reduced design space. Useful to know how close this gets to the theoretical optimum, probably much more preferable than arbitrarily weird sampling
- We consider either polyrhythmic or sequential uniform designs. 
- Reduces setup to a family of one-dimensional optimization problem. Family indexed by partition of measurement budget $N=N_1+N_2$. Once you choose partition, it's one-dimensional bounded optimization
- Solve full convex problem for comparison



Results 3: Simplest use of replicates
Here we present derivation of our main theorem


Results 4: Derivation of method

Discussion:

- Add on more sophisticated idea of biological vs technical replicates and how they tell you different things.









































